\documentclass[
  a4paper,               % A4-Papier
  twoside,               % zweiseitig
  DIV=12,                % Zwölfteilung der Seite
  BCOR=8mm,              % 8mm Korrektur für Bindung
  headinclude=true,      % Header hinzufügen
  footinclude=false,     % Footer nicht hinzufügen
  numbers=noenddot,      % keine Punkte nach Kapitelnummern
  headheight=40pt,       % Höhe des Header-Blocks
  11pt]{scrartcl}        % Schriftgröße

\usepackage[T1]{fontenc}
\usepackage[utf8]{luainputenc}
% not in use. use XeLaTeX compiler and fontspec if necessary.
%\usepackage[math-style=TeX, bold-style=TeX]{unicode-math}
\usepackage[ngerman]{babel}
\usepackage{amsfonts, amsmath, amsthm}
\usepackage{graphicx}
\usepackage{textcomp}
\usepackage{gensymb}
\usepackage{hyperref}
\usepackage{wrapfig}
\usepackage{xcolor}
\usepackage{todo}
\usepackage{booktabs}
\usepackage{pdfpages}
\xdefinecolor{unifarbe}{RGB}{0,75,90}

\usepackage{sectsty}
\chapterfont{\color{unifarbe}}  % sets colour of chapters
\sectionfont{\color{unifarbe}}  % sets colour of sections
\subsectionfont{\color{unifarbe}}  % sets colour of subsections
\begin{document}

\begin{titlepage}

\flushleft\includegraphics[width=0.5\textwidth]{./Logo_Uni_Luebeck_CMYK} \\[1cm]
\color{unifarbe}
\begin{center}
\textbf{\Huge Bachelorprojekt} \\[0.5cm]
\textbf{\Huge Optimierung biologisch-realistischer Neuronenmodelle}\\[0.5cm]

Institut für Robotik und Kognitive Systeme\\[2.5cm]
\textbf{\Large Dokumentation}\\[1.5cm]

Szymon Bereziak\\
Moritz Dannehl\\
Chris Girth\\
Can Kalelioglu\\
Julian Wolff\\
\end{center}

\vfill
\today

\end{titlepage}
\tableofcontents\newpage
\section{Einführung}
\subsection{Ziel des Projektes}

Mithilfe von künstlichen neuronalen Netzen wird versucht, Strukturen des menschlichen Gehirns zu simulieren. Die verwendeten Simulationen basieren dabei auf sehr vielen Parametern, deren Optimierung eine sehr langwierige Aufgabe darstellt. Es wurde im Rahmen dieses Projektes ein Framework erstellt, welches als Schnittstelle zwischen solchen Simulationen und Optimierungsalgorithmen fungiert. Dabei waren vor allem eine Modularisierung zwecks Erweiterbarkeit sowie eine einfache Bedienbarkeit über eine grafische Oberfläche Zentrum der Entwicklung.
\section{Features}
\begin{itemize}
\item einfache Anbindung eines (neuen) Frontends möglich durch ein simples Nachrichtenprotokoll
\item einfaches Hinzufügen von neuen Algorithmen
\item Modularisierung erlaubt komplettes Austauschen einzelner Module ohne die Kernlogik zu kennen
\item verteiltes Arbeiten möglich durch Anbindung des Simulationsclusters über SSH oder andere Protokolle
\item Multithreading bereits nativ in Kernlogik implementiert; Kern verwaltet verschiedene Läufe von Algorithmen selbstständig
\item kein aktives Warten notwendig; bei Terminierung eines Algorithmus' wird das Frontend automatisch benachrichtigt

\end{itemize}


\section{Architektur}
Das Framework ist in zwei Teile gegliedert. Der \emph{Kern} verwaltet selbstständig alle Algorithmen und spricht die Simulation an. Die Kommunikation zwischen dem Kern und der \emph{Benutzerschnittstelle} erfolgt über eine nachrichtenbasierte \emph{Kommunikationsschnittstelle}. Die Benutzerschnittstelle kann eine grafische Benutzeroberfläche oder ein Kommandozeileninterface sein. Über die Kommunikationsschnittstelle werden dem Kern Befehle erteilt, die das Einladen von Konfigurationen, Setzen von Parametern, Starten und Stoppen von Algorithmen sowie Statusabfragen umfassen.
\subsection{Kern}
Der Kern besteht aus den folgenden Komponenten:\\
	\begin{samepage}
	\textbf{main}. Verarbeitet eingehende Befehle von der Benutzerschnittstelle.\\
	\textbf{net}. Führt über eine SSH-Verbindung ein neuronales Netz und die zugehörige Analyse aus.\\
	\textbf{algorithms}. Stellt Optimierungsalgorithmen bereit. 
	\end{itemize}
	\end{samepage}
	\mbox{}\\
	
	\includegraphics[scale=0.55]{../presentation/sequenz.png}

\subsection{GUI}
% vim:ft=tex
\[
    \huge\mathfrak{Here,\ be\ dragons!}
\]

\section{Benutzerhandbuch}
\subsection{Ausführen der Anwendung}
Zur Ausführung sind die Pakete \texttt{python3} und \texttt{sshpass} notwendig.\\
Die Anwendung kann als Kommandozeilenanwendung mithilfe des Befehls \texttt{python3 main.py} oder alternativ mit grafischer Oberfläche mithilfe des Befehls \texttt{python3 main.py -{}-gui} ausgeführt werden.
\subsection{CLI}
Wird die Anwendung als Kommandozeilenanwendung gestartet, so kann sie mittels der folgenden Befehle gesteuert werden:\\
\begin{tabular}{p{0.4\linewidth}|p{0.55\linewidth}}
	\toprule
	Befehl & Auswirkung\tabularnewline
	\midrule
	\texttt{help} & Liefert eine Liste möglicher Befehle, ähnlich dieser
	Tabelle\tabularnewline
	\texttt{get\ algorithms} & liefert eine Liste von implementierten
		Algorithmen\tabularnewline
	\texttt{get\ algorithms\ \textless{}name\textgreater{}} & liefert die
	möglichen Parameter für den angegebenen Algorithmus\tabularnewline
	\texttt{get\ config} & liefert alle in der Konfigurationsdatei
	vorhandenen Sektionen\tabularnewline
	\texttt{get\ config\ \textless{}sec\textgreater{}} & liefert alle
	Optionen unter der gegebenen Sektion\tabularnewline
	\texttt{get\ config\ \textless{}sec\textgreater{}\ \textless{}opt\textgreater{}}
	& liefert den Wert der angegebenen Option\tabularnewline
	\texttt{set\ config\ \textless{}sec\textgreater{}\ \textless{}opt\textgreater{}\ \textless{}val\textgreater{}}
	& Setzt die angegebene Option auf den angegebenen Wert\tabularnewline
	\texttt{set\ password} & Öffnet eine Passworteingabe zur Eingabe des
	Passworts der SSH-Verbindung zum neuronalen Netz\tabularnewline
	\texttt{save\ config} & Speichert die Änderungen in der
	Konfigurationsdatei\tabularnewline
	\texttt{start\ \textless{}algorithm\textgreater{}\ \textless{}params...\textgreater{}}
	& Startet einen Optimierungsalgorithmus mit den gegebenen
	Parametern\tabularnewline
	\bottomrule
\end{tabular}\\
\newline\\
Bevor eine Optimierung gestartet wird sollten folgende Befehle
aufgerufen werden:\\
\begin{tabular}{p{0.4\linewidth}|p{0.55\linewidth}}
	\toprule
	Befehl & Auswirkung\tabularnewline
	\midrule
	\texttt{set\ password} & Setzt das SSH Passwort\tabularnewline
	\texttt{set\ config\ SSH\ host\ \textless{}user@ip\textgreater{}} & Setzt
	die Serveradresse\tabularnewline
	\texttt{set\ config\ SSH\ net\ \textless{}cmd\textgreater{}} & Setzt
	den Befehl zur Ausführung des neuronalen Netzes\tabularnewline
	\texttt{set\ config\ SSH\ analysis\ \textless{}cmd\textgreater{}} &
	Setzt den Befehl zur Ausführung der Analyse\tabularnewline
	\bottomrule
\end{tabular}\\
\newline\\
Anschließend können beliebige Optimierungsalgorithmen gestartet werden.\\
\newline\\
\begin{samepage}
Beispiel:\\
\texttt{
\noindent\hspace*{10mm}	>set config SSH host "bachelor1@localhost"\\
\noindent\hspace*{10mm}	>set config SSH net "cd  \raisebox{-0.6ex}{\~{ }}/acnet2 \&\& genesis acnet2.g"\\
\noindent\hspace*{10mm}	>set config SSH analysis "cd  \raisebox{-0.6ex}{\~{ }}/acnet2 \&\& python ./analysis.py"\\
\noindent\hspace*{10mm}	>save config\\
\noindent\hspace*{10mm}	>set password\\
\noindent\hspace*{10mm}	>start random\_search 4
	}\\
\end{samepage}
\subsection{GUI}
Wird das Programm mittels des Befehls
\begin{center}
\texttt{python3 main.py -{}-gui}
\end{center}
gestartet, öffnet sich das Hauptfenster der grafischen Benutzeroberfläche.
Abbildung 1 zeigt schematisch die Möglichkeiten, welche die GUI bietet:\\
\begin{figure}[h]
\includegraphics[scale = 2.5]{../presentation/GUI-use-case.png}
\caption{Use-case Diagramm GUI}
\end{figure}
\newpage
\begin{samepage}
Die folgende Tabelle liefert eine kurze Beschreibung der Funktionen:\\[0.5cm]
\begin{tabular}{p{0.4\linewidth}|p{0.55\linewidth}}
	\toprule
	Bezeichner & Funktion\tabularnewline
	\midrule
	\texttt{load session} & stellt die Daten einer zuvor gespeicherten Sitzung wieder her.\tabularnewline
	\texttt{add\ algorithm} & öffnet das addframe\tabularnewline
	\texttt{choose\ algorithm} & zeigt eine Auswahl der möglichen Algorithmen, von denen eine zu wählen ist.\tabularnewline
	\texttt{set\ params} & ermöglicht die Einstellung algorithmenspezifischer Parameter.\tabularnewline
	\texttt{view\ runtime} & liefert einen Überblick über die momentan gewählten Algorithmen und deren gegenwärtigen Status\tabularnewline
	\texttt{remove} & entfernt den ausgewählten Algorithmus\tabularnewline
	\texttt{run} & startet die Berechnung\tabularnewline
	\texttt{set\ ssh\ Keys} & Öffnet das Eingabefenster zur Eingabe des
	Passworts und anderer benötigter Parameter der SSH-Verbindung\tabularnewline
	\texttt{stop} & stoppt die Berechnung\tabularnewline
	\texttt{abort} & verwirft die Berechnung\tabularnewline
	\texttt{export data} & speichert die aktuelle Sitzung\tabularnewline
	\bottomrule
\end{tabular}\\
\end{samepage}

\subsubsection{SSH-Verbindung}
Direkt nach dem Aufruf der Applikation erscheint neben dem Hauptfenster das Fenster \texttt{ssh-Gate} in welchem die Credentials und Einstellungen der SSH-Verbindung.
\begin{figure}[h]
\includegraphics[scale=1]{ssh-frame.png}
\caption{Fenster zur Einstellung der SSH-Verbindung}
\end{figure}

Schlägt die Verbindung fehl, öffnet sich dieses Fenster erneut.

Durch clicken auf \textbf{Abbrechen} wird der dieses Fenster geschlossen, es können dann die Sessions weiter bearbeitet werden.

\subsubsection{Algorithmen hinzufügen}
Durch clicken auf \textbf{add algorithm} im Hauptfenster, erscheint das Menü zum Hinzufügen eines Algorithmus-Runs.
\begin{figure}[h]
\includegraphics[scale=1]{addalg.png}
\caption{Fenster zur Hinzufügen eines Algorithmus-Runs zur Session}
\end{figure}

Nach einem klick auf \textbf{adopt} werden die Einstellungen zur Session hinzugefügt.

Im linken Pull-Down-Menü sind die vorhandenen Algorithmen auswählbar.

\newpage
\subsubsection{Session Starten / Abbrechen}
Sind Algorithmen nun in der Session hinzugefügt, kann die Session nun über den orangenen Button \textbf{Run} gestartet werden.

\begin{figure}[h]
\includegraphics[scale=.45]{mainframe.png}
\caption{Fenster zur Hinzufügen eines Algorithmus-Runs zur Session}
\end{figure}


Eine Zeile in der Session-Tabelle ist eine Berechnung.

Läuf die Berechnung, ändert sich der button zu \textbf{STOP}, über welchen die laufende Session abgebrochen werden kann.

Es ist möglich, eine teilweise berechnete Session wieder aufzunehmen (weiterer Click auf \textbf{Run}).

\subsubsection{Session-Tabelle}
Die Tabelle gliedert sich in vier spalten, "ID", "Algorithm", "Status" und "Parameters".

\begin{description}
\item[ID] is eine fortlaufende Zahl. \\

\item[Algorithm] enthält den Namen des Algorihmuses.

\item[Status] zeigt den Status dieser Berechnung an. Wobei "stand-by" ein Algorithmus ist, der noch nicht berechnet wurde, "computing..." bedeutet die Berechnung findet gerade statt, "aborted" bedeutet die Berechnung wurde abgebrochen. Wurde eine Berechnung beended, befindet sich hier die Ausgabe.

\item[Parameters] enthält die Parameter die für die gegebene Berechnung ausgewählt wurden.
\end{description}

Bei einem Rechtsclick auf eine Tabellenzeile öffnet sich ein Kontextmenü mit den Optionen einen Algorithmus aus der Liste zu entfernen, einen abgebrochenen Algorithmus zu resetten und somit bei einem weiteren Durchgang wieder zu berechnen. Wird die Berechnung gerade durchgeführt, kann man diese einzeln abbrechen, die restliche Session aber nicht. Eine weitere Berechnung findet statt.

\subsubsection{Session laden / speichern}
Ein Click auf \textbf{load session} erlaubt es eine vorher abgespeicherte Session in die Tabelle zu laden. Bestehende Tabelleninhalte werden dabei gelöscht!
Es ist hierbei unerheblich, ob die Berechnungen bereits beendet sind, abgebrochen wurden, oder noch nicht durchgeführt worden sind.
\vspace{\baselineskip}

Ein Click auf \textbf{Export Data} erlaubt es diese Session abzuspeichern.
Eine exportierte Session kann hinterher über \textbf{load session} wieder geladen werden. \\
Die Daten werden als einfache CSV-Datei abgespeichert und können problemlos in Textverarbeitungsprogrammen genutzt werden.

\end{document}
