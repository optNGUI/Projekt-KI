\documentclass[
  a4paper,               % A4-Papier
  twoside,               % zweiseitig
  DIV=12,                % Zwölfteilung der Seite
  BCOR=8mm,              % 8mm Korrektur für Bindung
  headinclude=true,      % Header hinzufügen
  footinclude=false,     % Footer nicht hinzufügen
  numbers=noenddot,      % keine Punkte nach Kapitelnummern
  headheight=40pt,       % Höhe des Header-Blocks
  11pt]{scrartcl}        % Schriftgröße

\usepackage[T1]{fontenc}
\usepackage[utf8]{luainputenc}
\usepackage[ngerman]{babel}
\usepackage{amsfonts, amsmath, amsthm}
\usepackage{graphicx}
\usepackage{textcomp}
\usepackage{gensymb}
\usepackage{hyperref}
\usepackage{wrapfig}
\usepackage{xcolor}
\xdefinecolor{unifarbe}{RGB}{0,75,90}

\usepackage{sectsty}
\chapterfont{\color{unifarbe}}  % sets colour of chapters
\sectionfont{\color{unifarbe}}  % sets colour of sections
\subsectionfont{\color{unifarbe}}  % sets colour of subsections
\begin{document}

\begin{titlepage}

\flushleft\includegraphics[width=0.5\textwidth]{./Logo_Uni_Luebeck_1200dpi}Korrektes Logo einsetzen!!!\\[1cm]
\color{unifarbe}
\begin{center}
\textbf{\Huge Bachelor Projekt} \\[0.5cm]
\textbf{\Huge Optimierung biologisch-realistischer Neuronenmodelle}\\[0.5cm]

Institut für Robotik und kognitive Systeme\\[2.5cm]
\textbf{\Large Dokumentation}\\[1.5cm]

Szymon Bereziak\\
Moritz Dannehl\\
Chris Girth\\
Can Kalelioglu\\
Julian Wolff\\
\end{center}

\vfill
\today

\end{titlepage}
\tableofcontents\newpage
\section{Einführung}
Hier steht irgendwann mal irgendwas.
\section{Features}

\section{Architektur}
\subsection{Kern}
Der Kern besteht aus den folgenden Komponenten:\\
	\begin{samepage}
	\textbf{main}. Verarbeitet eingehende Befehle von der Benutzerschnittstelle.\\
	\textbf{net}. Führt über eine SSH-Verbindung ein neuronales Netz und die zugehörige Analyse aus.\\
	\textbf{algorithms}. Stellt Optimierungsalgorithmen bereit. 
	\end{itemize}
	\end{samepage}
	\mbox{}\\
	
	\includegraphics[scale=0.55]{../presentation/sequenz.png}

\subsection{GUI}
% vim:ft=tex
\[
    \huge\mathfrak{Here,\ be\ dragons!}
\]

\section{Benutzerhandbuch}
\subsection{Erste Untersektion}
\end{document}